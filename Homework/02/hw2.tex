\documentclass[10pt]{article}

\usepackage{graphicx}
\usepackage[tableposition=top]{caption}

\usepackage{floatrow}
\floatsetup[table]{capposition=top}

\usepackage{fancyhdr}
% \usepackage[margin=1in,showframe]{geometry}
\usepackage[margin=1in]{geometry}
\usepackage{tabularx} % for \textwidth table
\usepackage{multicol}
\usepackage{hyperref}
\hypersetup{
  colorlinks,
  urlcolor=blue,
  pdftitle={DM Curriculum Vitae},
  pdfauthor={Dylan Mikesell},
  pdfsubject={Curriculum Vitae},
}

\pagestyle{fancy}

\lhead{Boise State University}
\chead{GEOS 397}
\rhead{Fall 2016}
\fancyfoot[C]{\thepage}
% \renewcommand{\headrulewidth}{0pt} % remove horizontal line
\setlength{\headsep}{12pt} % set separation between course name and header
% \setlength{\hoffset}{-0.5in} % set separation between course name and header
% \setlength{\textwidth}{7in}

\setlength\parindent{0pt} % Removes all indentation from paragraphs
\setlength{\parskip}{\baselineskip}

\title{Computation in the geosciences}

\date{\empty}

% \renewcommand{\chaptertitle}[1]{\chaptitlefont\MakeUppercase{#1}}

\begin{document}

% make the title
\begin{multicols}{2}
\begin{flushleft}
\textbf{Homework \#2: Data types, algebra, plotting and loops}
\end{flushleft}
\columnbreak 
\begin{flushright}
\textbf{Due: 5:00 PM 09/09/16}
\end{flushright}
\end{multicols}

Please read the following questions carefully and make sure to answer the Parts completely. In your MATLAB script(s), please include these questions and part numbers with your answers. Then use the \textit{Publish} function in MATLAB to publish your script to a \textit{pdf} document. For more on the Publish functionality within MATLAB see \url{http://www.mathworks.com/help/matlab/matlab_prog/publishing-matlab-code.html}. See the last page for instructions on uploading your document to blackboard.

\subsection*{Preliminaries}

Use comments to document your code, and distinguish the different \textit{Problems} using 'Sections'. I suggest that you review two MathWorks, Inc. webpages.
\begin{itemize}
	\item \href{http://www.mathworks.com/help/matlab/matlab_prog/improve-code-readability.html}{Improving code readability}
	\item \href{http://www.mathworks.com/help/matlab/matlab_prog/run-sections-of-programs.html}{Using 'Sections' in your coding}
\end{itemize}

The goal of this homework set is to familiarize you with some basic MATALB variables and functions. Variable names are in \textit{italics} and MATLAB functions are in \textbf{bold}. Remember that you can type \textbf{help 'function'} to get the MATLAB help for a given function. Make sure to answer all subquestions. The easiest way to do this is by including answers in your MATLAB script and using the Publish button on the \textbf{Publish} tab within MATLAB.

%----------------------------------------------------------------------------
\section*{Problem 1: Variables and time \textit{(20 pts.)}}
%----------------------------------------------------------------------------

A) Create a variable \textit{time} using the \textbf{clock} function. Answer the following:
\begin{itemize}
	\item What is the size of \textit{time}?
	\item Is it a row or column vector?
	\item What does \textit{time} contain? (be specific)
	\item What variable class is \textit{time}?
\end{itemize}

B) Use \textbf{datestr} to make a new variable called \textit{yearString} that is only the year. \textit{Hint: you may need to read the \textbf{help datestr.}}

C) Save \textit{time} and \textit{yearString} in a '.mat' file. Call this file whatever you like. Use the \textbf{save} function in MATLAB.

FYI: \textbf{datenum} is another useful MATLAB function that is worth looking into when you have time.

\pagebreak
%----------------------------------------------------------------------------
\section*{Problem 2: Plotting sine and cosine waves \textit{(40 pts.)}}
%----------------------------------------------------------------------------

A) Create a vector using \textbf{linspace} that goes from tMin=0 to tMax=1~second. Call this vector \textit{tArray} and make sure it has 1001 elements. Answer the following:
\begin{itemize}
	\item What is the sample interval (in seconds) of \textit{tArray}?
\end{itemize}

B) Make a variable \textit{f} and set it equal to 5~Hz. 

C) How do you convert \textit{f} to angular frequency? Do this and call the new variable $\omega$.

D) Compute the cosine and sine of \textit{$\omega$*tArray} and assign each of these to a variable. \textit{Hint: Make sure you use the built-in MATLAB sine and cosine functions.} 

E) Plot these two curves on the same plot as a function of \textit{tArray} (i.e. time on the x-axis). \textit{Hint: you may need to look at the \textbf{plot} and \textbf{hold} commands.} Plot the cosine function in red with a solid line and plot the sine function in blue with a dashed line. \textit{Hint: use \textbf{help plot} and search the online documentation for MATLAB plotting help}. Make sure to label the $x$ and $y$ axes.

F) Use \textbf{legend} to add a legend outside of the plot window in the upper-right corner of the figure. 

G) Use \textbf{axis} to set the x-axis limits to [tMin, tMax] and the y-axis limits to [-1.5, 1.5]. 

H) Use \textbf{grid} to turn on a background grid in the plot.

I) Starting from t=0, determine if you did in fact compute 5~Hz sine and cosine waves. Discuss your observations of the differences and similarities between the sine and cosine waves.

J) What is the amplitude of the waves?

K) How could you change the amplitude?

\pagebreak
%----------------------------------------------------------------------------
\section*{Part 3: Population growth \textit{(40 pts.)}}
%----------------------------------------------------------------------------

\subsection*{Preliminaries}

A typical computer program follows this logical sequence:
\begin{enumerate}
	\item Define parameters and constants
	\item Create the model space (includes time if necessary)
	\item Computing calculations (using loops if necessary)
	\item Plot results
\end{enumerate}

In this problem, the model will be a study of rabbit population growth based on a growth factor per unit time. The easiest way to visualize a population growth model is to assume that some proportion of the population reproduces during each time step and some proportion dies. The difference between the two is the growth factor. The growth factor therefore must be a number between 0 and 1. If the factor is 0 then the population does not grow, as the number of rabbits born equals the number of rabbits that die in a time step. If the factor is 1 then the population doubles at every time step (i.e. gestation period)

The differential equation representing this system is 
\begin{equation}
	\frac{dR}{dt} = b*R.
	\label{eqn:diffEQ}
\end{equation}
This says that the change in rabbits with time ($dR/dt$) is equal to the growth rate ($b$) times the population ($R$). We can solve this differential equation for the total population at a given time as follows:
\begin{eqnarray}
	dR &=& b*R*dt \\
	dR &=& R_n-R_o = b*R*dt \\
	R_n &=& R_o + b*R*dt.
\end{eqnarray}
$R_n$ is the total number of rabbits (i.e. population) given the number of rabbits initially ($R_o$) for a time step $dt$ and a growth factor ($b$). A simple description of this model is \textit{Have = Had + Change}. Our next goal is to implement this model in our MATLAB script.



\vfill

Upload your \textit{pdf} file to the blackboard under Assignment \#2. Your filename should be \textit{GEOS397\_HW2\_Lastname.pdf}. Hint: You can achieve this automatically by calling your MATLAB script \textit{GEOS397\_HW2\_Lastname.m}.

\end{document}
