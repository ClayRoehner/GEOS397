\documentclass[10pt]{article}

\usepackage{graphicx}
\usepackage[tableposition=top]{caption}

\usepackage{floatrow}
\floatsetup[table]{capposition=top}

\usepackage{fancyhdr}
% \usepackage[margin=1in,showframe]{geometry}
\usepackage[margin=1in]{geometry}
\usepackage{tabularx} % for \textwidth table
\usepackage{multicol}
\usepackage{hyperref}
\hypersetup{
  colorlinks,
  urlcolor=blue,
  pdftitle={DM Curriculum Vitae},
  pdfauthor={Dylan Mikesell},
  pdfsubject={Curriculum Vitae},
}

\pagestyle{fancy}

\lhead{Boise State University}
\chead{GEOS 397}
\rhead{Fall 2016}
\fancyfoot[C]{\thepage}
% \renewcommand{\headrulewidth}{0pt} % remove horizontal line
\setlength{\headsep}{12pt} % set separation between course name and header
% \setlength{\hoffset}{-0.5in} % set separation between course name and header
% \setlength{\textwidth}{7in}
\setlength\parindent{0pt} % Removes all indentation from paragraphs

\title{Computation in the geosciences}

\date{\empty}

% \renewcommand{\chaptertitle}[1]{\chaptitlefont\MakeUppercase{#1}}

\begin{document}

% make the title
\begin{multicols}{2}
\begin{flushleft}
\textbf{Homework \#1: GIT, Markdown and MATLAB variables}
\end{flushleft}
\columnbreak 
\begin{flushright}
\textbf{Due: 5:00 PM 09/02/16}
\end{flushright}
\end{multicols}

Please read the following questions carefully and make sure to answer the Parts completely. In your MATLAB script, please include these questions and part numbers with your answers. Then use the \textit{Publish} function in MATLAB to publish your script to a \textit{pdf} document. See bottom of page for instructions on uploading your document to blackboard. We have decided to use pdf format because blackboard has problems with html.

\section*{Part 1 \textit{(30 pts.)}}

\noindent Make a Github account using your @u.boisestate.edu email address. Then, using the Github Desktop app, clone the \textit{master} branch of the GEOS397 project to your local directory. Make a new branch called GEOS397\_Lastname, where you insert your last name.

\section*{Part 2 \textit{(30 pts.)}}

\noindent In your new branch, make an new file in the HW1 directory called \textit{GEOS397\_HW1\_Lastname.m}. 

\noindent Use a MATLAB script to write a summary of how you would go about ensuring that (if the clas had 10 students) you would partner with every other student for the 9 homework sets (you can write some equations if you want). (Hint: Use the publish tab on the top ribbon to help format the document.) Keep in mind that a constraint imposed on this problem is that no two students in the class can have repeat partners.

\section*{Part 3 \textit{(20 pts.)}}

\noindent In the same file, list all of the possible variable types in MATLAB that are covered in the MATLAB style guide reading assignment. Also, give a description of each type and list why this is a useful type of variable.

\section*{Part 4 \textit{(20 pts.)}}

\noindent Based on the reading \textit{MatlabStyle1p5.pdf}, give an example variable name for each of the variable types you identified in Part 3. Then publish your MATLAB script as a pdf file; also \textit{commit} your changes to your specific GIT branch; DO NOT publish though.

\vspace{2\baselineskip}
\vfill

Upload your \textit{pdf} file to the blackboard under Assignment \#1. Your filename should be \textit{GEOS397\_HW1\_Lastname.pdf}. Hint: You can achieve this automatically by calling your MATLAB script \textit{GEOS397\_HW1\_Lastname.m}.

\end{document}
